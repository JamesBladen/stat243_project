\documentclass[a4paper]{book}
\usepackage[times,inconsolata,hyper]{Rd}
\usepackage{makeidx}
\usepackage[latin1]{inputenc} % @SET ENCODING@
% \usepackage{graphicx} % @USE GRAPHICX@
\makeindex{}
\begin{document}
\chapter*{}
\begin{center}
{\textbf{\huge ARS Manual}}
\par\bigskip{\large \today}
\end{center}
\inputencoding{utf8}
\HeaderA{ARSpackage}{ARSpackage: an Adaptive Rejection Sampler}{ARSpackage}
\keyword{package, rejection sampling}{ARSpackage}
%
\begin{Description}\relax
Final project for Statistics 243, an R package that performs adaptive rejection sampling, first proposed by Gilks and Wild in 1992.
\end{Description}
%
\begin{Details}\relax

\Tabular{ll}{
Package: & ARSpackage\\{}
Type: & Package\\{}
Version: & 1.0\\{}
Date: & 2013-12-13\\{}
Depends: & methods, numDeriv\\{}
Collate: & 'adapt\_reject.r', 'ars\_methods.r'
}
\end{Details}
%
\begin{Author}\relax
J. Bladen, L. Felberg, H.W. Tsao, S. Tu
\end{Author}
%
\begin{References}\relax
Gilks, Wild, 1992. \url{http://faculty.chicagobooth.edu/hedibert.lopes/teaching/ccis2010/1992GilksWild.pdf}.
\end{References}
%
\begin{SeeAlso}\relax
\url{https://bitbucket.org/lfelberg/stat243_final_proj} \\{}
\url{https://github.com/paciorek/stat243-fall-2013/tree/master/project}
\end{SeeAlso}
%
\begin{Examples}
\begin{ExampleCode}
	# Testing the normal distribution from -Inf to Inf
	n_samples <- ars( 10000, fx = function(x){(1/sqrt(2*pi)*exp((-(x-0)^2)/2))}, bounds=c(-Inf, Inf) )

	# Testing Gamma(2,1) on interval[0.01,Inf]
	sample<-ars(10000,function(x){1/2*x*exp(-x)},c(0.01, Inf))
\end{ExampleCode}
\end{Examples}
\inputencoding{utf8}
\HeaderA{Cadapt\_reject\_sample}{The adapt\_reject class}{Cadapt.Rul.reject.Rul.sample}
%
\begin{Description}\relax
This class contains all the methods used to perform an AR
sampling.
\end{Description}
%
\begin{Section}{Slots}
\begin{description}
 \item[\code{n}:] Variable of class
\code{"numeric"}, n, containing the number of points to
sample, taken as user input.\item[\code{f\_x}:] Function
of class \code{"function"}, containing the f(x) to sample
from, taken as user input.
\item[\code{bounds}:] Variable of class \code{"numeric"},
n, containing the bounds of the function, taken as user
input.\item[\code{output}:] Variable of class
\code{"vector"}, containing sampled points to return to
user.\item[\code{h\_at\_x}:] Variable of class
\code{"vector"}, containing computed log(f(x)) values at
all x values\item[\code{hprime\_at\_x}:] Variable of
class \code{"vector"}, containing computed derivative of
log(f(x)) values at all x values
\item[\code{z}:] Variable of class \code{"vector"},
containing abscissae of upper bound function.
\item[\code{samples}:] Variable of class \code{"vector"},
containing random numbers generated by s(x) and unif.
\item[\code{x}:] Variable of class \code{"vector"},
containing x values used in ARS.
\item[\code{weights}:] Variable of class \code{"vector"},
containing sampled points to return to user.
\item[\code{output}:] Variable of class \code{"numeric"},
containing sampled points to return to user.
\item[\code{mat\_sorted}:] Variable of class
\code{"matrix"}, containing x values, their corresponding
h and h prime values, sorted by increasing x.
\item[\code{guess\_of\_mode}:] Variable of class
\code{"numeric"}, containing an optional user input guess
of the mode of the distribution, should be within 200 of
actual mode.
\end{description}

\end{Section}
\inputencoding{utf8}
\HeaderA{Cadapt\_reject\_sample}{The adapt\_reject class}{Cadapt.Rul.reject.Rul.sample}
\aliasA{ev\_h}{Cadapt\_reject\_sample}{ev.Rul.h}
\aliasA{gen\_x}{Cadapt\_reject\_sample}{gen.Rul.x}
\aliasA{initialize}{Cadapt\_reject\_sample}{initialize}
\aliasA{lower}{Cadapt\_reject\_sample}{lower}
\aliasA{sampling}{Cadapt\_reject\_sample}{sampling}
\aliasA{s\_x}{Cadapt\_reject\_sample}{s.Rul.x}
\aliasA{update}{Cadapt\_reject\_sample}{update}
\aliasA{upper}{Cadapt\_reject\_sample}{upper}
\aliasA{validity\_ars}{Cadapt\_reject\_sample}{validity.Rul.ars}
%
\begin{Description}\relax
This class contains all the methods used to perform an AR
sampling.

A method to intialize the ARS class for sampling.  Will
store values input from user and will also initialize
empty arrays for all other slots.

The main objective of this validity check is to ensure at
creation that the number of samples desired is a positive
integer

Cadapt\_reject\_sample method for generating first two
points.  If the distribution is unbounded, then find the
function's mode and pick points surrounding it.  If it's
bounded on one side, we use the bound given and search
until we find a point that corresponds to the opposite
end of the domain with respect to their derivatives.  If
bounded on both sides, use given bounds.

Cadapt\_reject\_sample method that evaluates the log(f(x))
for a given x and the derivative as well.

Cadapt\_reject\_sample method to normalize the upper bounds
of log(f(x)).  Multiple objective are performed here. The
most important being the calculation of the abcissa
vector Z.  Additionally, the weights and exact values of
the piecewise integration of each interval and the
normalization factor for the entire upper bound are
calculated and the x's, their evaluations and their
derivatives are sorted by x.

Method to sample from s\_x.  The basic algorithm is as
follows: 1. Determine an interval to sample from using
the weights of integration of the function on each
interval, computed in the s\_x method. 2.  Use inverse CDF
method to sample from within the selected interval.
Return the object with new sample.

Cadapt\_reject\_sample method to evaluate the upper bound
of x\_star.

Cadapt\_reject\_sample method to evaluate the lower bound
of x\_star.

Cadapt\_reject\_sample method to determine which ACC/REJ
criteria a given sampled value fits into and updates the
samples and x values accordingly.
\end{Description}
%
\begin{Usage}
\begin{verbatim}
  validity_ars(object)
\end{verbatim}
\end{Usage}
%
\begin{Arguments}
\begin{ldescription}
\item[\code{object}] \code{\LinkA{Cadapt\_reject\_sample}{Cadapt.Rul.reject.Rul.sample.Rdash.class}}
object

\item[\code{n}] \code{numeric} determining the number of samples
to obtain

\item[\code{f\_x}] \code{function} for distribution to sample
from

\item[\code{bounds}] \code{vector} of distribution bounds

\item[\code{guess\_of\_mode}] \code{numeric} optional idea of
where distribution is located \#

\item[\code{object}] An \code{adapt\_reject\_sample} object \#

\item[\code{object}] \code{\LinkA{Cadapt\_reject\_sample}{Cadapt.Rul.reject.Rul.sample.Rdash.class}}
object \#

\item[\code{object}] \code{\LinkA{Cadapt\_reject\_sample}{Cadapt.Rul.reject.Rul.sample.Rdash.class}}
object \#

\item[\code{object}] \code{\LinkA{Cadapt\_reject\_sample}{Cadapt.Rul.reject.Rul.sample.Rdash.class}}
object \#

\item[\code{object}] \code{\LinkA{Cadapt\_reject\_sample}{Cadapt.Rul.reject.Rul.sample.Rdash.class}}
object \#

\item[\code{object}] \code{\LinkA{Cadapt\_reject\_sample}{Cadapt.Rul.reject.Rul.sample.Rdash.class}}
object \#

\item[\code{object}] \code{\LinkA{Cadapt\_reject\_sample}{Cadapt.Rul.reject.Rul.sample.Rdash.class}}
object \#

\item[\code{object}] \code{\LinkA{Cadapt\_reject\_sample}{Cadapt.Rul.reject.Rul.sample.Rdash.class}}
object \#
\end{ldescription}
\end{Arguments}
%
\begin{Section}{Slots}
\begin{description}
 \item[\code{n}:] Variable of class
\code{"numeric"}, n, containing the number of points to
sample, taken as user input.\item[\code{f\_x}:] Function
of class \code{"function"}, containing the f(x) to sample
from, taken as user input.
\item[\code{bounds}:] Variable of class \code{"numeric"},
n, containing the bounds of the function, taken as user
input.\item[\code{output}:] Variable of class
\code{"vector"}, containing sampled points to return to
user.\item[\code{h\_at\_x}:] Variable of class
\code{"vector"}, containing computed log(f(x)) values at
all x values\item[\code{hprime\_at\_x}:] Variable of
class \code{"vector"}, containing computed derivative of
log(f(x)) values at all x values
\item[\code{z}:] Variable of class \code{"vector"},
containing abscissae of upper bound function.
\item[\code{samples}:] Variable of class \code{"vector"},
containing random numbers generated by s(x) and unif.
\item[\code{x}:] Variable of class \code{"vector"},
containing x values used in ARS.
\item[\code{weights}:] Variable of class \code{"vector"},
containing sampled points to return to user.
\item[\code{output}:] Variable of class \code{"numeric"},
containing sampled points to return to user.
\item[\code{mat\_sorted}:] Variable of class
\code{"matrix"}, containing x values, their corresponding
h and h prime values, sorted by increasing x.
\item[\code{guess\_of\_mode}:] Variable of class
\code{"numeric"}, containing an optional user input guess
of the mode of the distribution, should be within 200 of
actual mode.
\end{description}

\end{Section}
\inputencoding{utf8}
\HeaderA{ars}{ars: The adapt\_reject function}{ars}
%
\begin{Description}\relax
This calls the class Cadapt\_reject\_sample and its
methods.  It returns a vector of samples generated via
the Adaptive rejective sampling method.
\end{Description}
%
\begin{Usage}
\begin{verbatim}
  ars(n_samples, fx, bounds = c(-Inf, Inf),
    guess_of_mode = -999, ...)
\end{verbatim}
\end{Usage}
%
\begin{Arguments}
\begin{ldescription}
\item[\code{n\_samples:}] Number of samples desired from
distribution

\item[\code{fx:}] Function to sample from

\item[\code{bounds:}] Bounds of function of interest.  The
default is an unbounded function
\end{ldescription}
\end{Arguments}
%
\begin{Value}
a vector containing \code{n} points sampled from the f(x)
distribution
\end{Value}

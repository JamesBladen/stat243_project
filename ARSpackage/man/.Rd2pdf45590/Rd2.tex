\documentclass[a4paper]{book}
\usepackage[times,inconsolata,hyper]{Rd}
\usepackage{makeidx}
\usepackage[latin1]{inputenc} % @SET ENCODING@
% \usepackage{graphicx} % @USE GRAPHICX@
\makeindex{}
\begin{document}
\chapter*{}
\begin{center}
{\textbf{\huge ARS Manual}}
\par\bigskip{\large \today}
\end{center}
\inputencoding{utf8}
\HeaderA{ARSpackage}{ARSpackage: an Adaptive Rejection Sampler}{ARSpackage}
\keyword{package, rejection sampling}{ARSpackage}
%
\begin{Description}\relax
Final project for Statistics 243, an R package that performs adaptive rejection sampling, first proposed by Gilks and Wild in 1992.
\end{Description}
%
\begin{Details}\relax

\Tabular{ll}{
Package: & ARSpackage\\{}
Type: & Package\\{}
Version: & 1.0\\{}
Date: & 2013-12-13\\{}
Depends: & methods, numDeriv\\{}
Collate: & 'adapt\_reject.r', 'ars\_methods.r'
}
\end{Details}
%
\begin{Author}\relax
J. Bladen, L. Felberg, H.W. Tsao, S. Tu
\end{Author}
%
\begin{References}\relax
Gilks, Wild, 1992. \url{http://faculty.chicagobooth.edu/hedibert.lopes/teaching/ccis2010/1992GilksWild.pdf}.
\end{References}
%
\begin{SeeAlso}\relax
\url{https://bitbucket.org/lfelberg/stat243_final_proj} \\{}
\url{https://github.com/paciorek/stat243-fall-2013/tree/master/project}
\end{SeeAlso}
%
\begin{Examples}
\begin{ExampleCode}
	# Testing the normal distribution from -Inf to Inf
	n_samples <- ars( 10000, fx = function(x){(1/sqrt(2*pi)*exp((-(x-0)^2)/2))}, bounds=c(-Inf, Inf) )

	# Testing Gamma(2,1) on interval[0.01,Inf]
	sample<-ars(10000,function(x){1/2*x*exp(-x)},c(0.01, Inf))
\end{ExampleCode}
\end{Examples}
\inputencoding{utf8}
\HeaderA{ars}{ars: The adapt\_reject function}{ars}
%
\begin{Description}\relax
This calls the class Cadapt\_reject\_sample and its
methods.  It returns a vector of samples generated via
the Adaptive rejective sampling method.
\end{Description}
%
\begin{Usage}
\begin{verbatim}
  ars(n_samples, fx, bounds = c(-Inf, Inf),
    guess_of_mode = 0, ...)
\end{verbatim}
\end{Usage}
%
\begin{Arguments}
\begin{ldescription}
\item[\code{n\_samples:}] Number of samples desired from
distribution

\item[\code{fx:}] Function to sample from

\item[\code{bounds:}] Bounds of function of interest.  The
default is an unbounded function
\end{ldescription}
\end{Arguments}
%
\begin{Value}
a vector containing \code{n} points sampled from the f(x)
distribution
\end{Value}

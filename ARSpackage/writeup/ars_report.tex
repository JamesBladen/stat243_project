\documentclass[11pt, oneside]{article}   	% use "amsart" instead of "article" for AMSLaTeX format
\usepackage[margin=.90in]{geometry}                		% See geometry.pdf to learn the layout options. There are lots.
\geometry{letterpaper}                   		% ... or a4paper or a5paper or ... 
\usepackage[parfill]{parskip}    		% Activate to begin paragraphs with an empty line rather than an indent
\usepackage{graphicx}				% Use pdf, png, jpg, or eps� with pdflatex; use eps in DVI mode
								% TeX will automatically convert eps --> pdf in pdflatex		
\usepackage{amssymb}
\usepackage{graphicx}
\usepackage{xcolor,cancel}
\usepackage{amsmath}
\usepackage{slashed}
\usepackage{listings}
\lstset{language=R}
%\usepackage[section]{placeins}

\newcommand\hcancel[2][black]{\setbox0=\hbox{$#2$}%
\rlap{\raisebox{.45\ht0}{\textcolor{#1}{\rule{\wd0}{1pt}}}}#2}

\usepackage{listings}             % Include the listings-package
\lstset{language=matlab}          % Set your language ( you can change the language for each code-block optionally )
\lstset{showspaces=false, showtabs=false, breaklines=true}

\title{Stat 243: Building an adaptive rejection sampler}
\author{James Bladen, Lisa Felberg, Siwei Tu, Hsin-Wei Tsao}
\date{December 13, 2013}							% Activate to display a given date or no date

\begin{document}
\maketitle

%%%%%%%%%%%%%%%

\section{Introduction}

\subsection*{ (i) }

Adaptive Rejection sampling was first introduced in 1992 by Gilks and Wild.  It was proposed as an alternative to vanilla rejection sampling for functions that are difficult to evaluate multiple times.  The objective of this method is to sample from a "difficult" distribution by representing it with tangent lines and evaluating the actual function as few times as possible.  The basic algorithm is as follows:

\begin{center}
	\begin{itemize}
		\item{ Take the log of the function to sample from, $h(x) = log(f(x))$ }
		\item{ Select two starting points, $x_1$ and $x_2$ that are to the right and to the left of the function's maximum, respectively}
		\item{ Sample two random numbers, one from the uniform random number distribution, URN, and one from the  normalized exponential of the upper bounded line, $x^*$ }
		\item{ If $URN \le exp(l(x*)-u(x*))$, accept and add $x^*$ to vector of $x$'s }
		\item{ Else if $URN \le exp(h(x*)-u(x*))$, accept and add $x^*$ to vector of $x$'s }
		\item{ Else reject }
		\item{ Continue to sample numbers and testing procedure until a desired number of samples is obtained}
	\end{itemize}
\end{center}

The image below is a graphical representation of the algorithm.

\begin{figure}[htbp!]
  \centering
  \caption{ Algorithm of ARS method}
    \includegraphics[width=.93\textwidth]{algorithm}
\end{figure}


%%%%%%%%%%%%%%%

\section{Code Structure}

We decided to utilize the S4 class as a format for our code.  It was chosen because of the formal structure and modularity.  These properties also made it easy to test and create in a very piecewise manner.  The general structure to the code is a follows:  A wrapper function \textit{a\_r\_s} runs the sampling method.  The program is run as follows:

 $$ ars \leftarrow a\_r\_s( \mathbf{n}=\{\# \, of \, samps \, desired \}, \mathbf{f\_x} = \{f(x) \, to \, sample\}, \mathbf{bounds}= \{ bds \, of \, f(x)\, \} ) $$
 
 This will return a \textit{Cadapt\_reject\_sample} class which contains sampled points in \textit{ars@samples}.  The class itself contains all the methods and variables required to perform adaptive rejection sampling.  
 
 First, two points are chosen as starting $x_1$ and $x_2$ values.  Next, $x^*$ values are drawn from the corresponding $u(x)$ distribution and the sampling criteria is tested.  This is repeated until the desired number of samples is obtained.    Brief descriptions of the most important methods are given below.  We've included more detailed source documentation as part of our solution as well (see \textit{a\_r\_sManual.pdf}).

\subsection*{ gen\_x method }
In this method, we generate two initial points i.e. $x_{1}$, $x_{2}$ according to user's input of bounds. \\
There are 4 possible cases:
  \begin{enumerate}
  \item  If the target function is bounded on both sides, then we take 2 bounds as $x_{1}$, $x_{2}$.
  \item  If the target function is bounded only on left side,  we take $x_{1}$ as the left bound, and find an $x$ on the right side of $x_{1}$ s.t. $h'(x_{1})h'(x)<0$, then take $x$ as $x_{2}$.
  \item  If the target function is bounded only on right side, we take  $x_{1}$ as the right bound, and find an $x$ on the left side of $x_{1}$ s.t. $h'(x_{1})h'(x)<0$, then take $x$ as  $x_{2}$.
  \item  If the target function is unbounded on both sides, we first set $x_{1}=0$. If $h'(0)>0$ then find an $x$ on the right side of $x_{1}$ s.t. $h'(x)<0$, then take $x$ as $x_{2}$. Similary, if $h'(0)<0$ then find an $x$ on the left side of $x_{1}$ s.t. $h'(x)>0$, then take $x$ as  $x_{2}$. In addition, if $h'(0)=0$ which means $h(x)$ is symmetric to $x=0$, then we set $(-\frac{1}{2},\frac{1}{2})$ to be 2 starting value.
 \end{enumerate}
Note that $h(x)=log(f(x))$ which is concave. So for any $x_1, x_2$, if $h'(x_1)h'(x_2)<0$, $x_1$ and $x_2$ would be on the different side of maximum of $h(x)$. And with the options of 2 to 4, we save $x$, $h(x)$ and $h'(x)$ for all $x$'s we computed in the process for later use.


\subsection*{ ev\_h method }
In this method, we calculate the values of $h(x)$ and $h'(x)$ for any $x$ with $genD(\,)$ in package $numDeriv$.

\subsection*{ s\_x method }
In this method, we calculate $s_{k}(x)$ by normalizing $e^{ u_{k}(x)}$. To normalize the $e^{ u_{k}(x)}$, we integrate each piece of $u(x)$ with following algorithm: \\
First note that $u(x)$ is a piecewise defined function. It is linear in each piece of interval. Actually for $x \in [z_{j-1}, z_j]$, $u(x) = ax+b $, where $a=h'(x_j)$ and $b=h(x_j)- x_j h'(x_j)$. \\
So for $x \in [z_{j-1}, z_j]$, 
\begin{displaymath}
\int_{z_{j-1}}^{z_j} \, e^{u(x)} dx =\left. \frac{1}{h'(x_j)} \, exp( h'(x_j)x + h(x_j) -x_j h'(x_j))  \right|_{z_{j-1}}^{z_j},
\end{displaymath}
and
\begin{displaymath}
\int_D e^{u_k(x')} dx' = \left.\sum_j \frac{1}{h'(x_j)} \,exp( h'(x_j)x + h(x_j) -x_j h'(x_j))  \right|_{z_{j-1}}^{z_j}.
\end{displaymath}

In the special case of $a=0$ which means that for $x \in [z_{j-1}, z_j]$, $u(x) = b $ where $b=h(x_j)$. \\Then
\begin{displaymath}
 \int_{z_{j-1}}^{z_j} \, e^{u(x)} dx =\left. e^bx_j\right|_{z_{j-1}}^{z_j} , 
\end{displaymath}
and, 
\begin{displaymath}
\int_D e^{u_k(x')} dx' =\left.\sum_j  e^bx_j\right|_{z_{j-1}}^{z_j}.
\end{displaymath}


\subsection*{ sampling method }
In this method, we sample a new $x^{*}$ from the $s_{k}(x)$ by using inverse CDF of  $s_{k}(x)$ with a random value $u \sim unif(0,1)$. Moreover, the inverse CDF algorithm we use is as follow: \\
Once we decided which interval $x^*$ falls in, we want to sample from $\frac{e^{u(x)}} {\int e^{u(x)} dx}$. Suppose $x^* \in [z_{j-1}, z_j]$, then $u(x) = ax+b $, where $a=h'(x_j)(\neq0)$ and $b=h(x_j)- x_j h'(x_j)$. \\
Let
\begin{displaymath}
C=\int_{z_{j-1}}^{z_j} \, e^{u(x)} dx,
\end{displaymath}
then the pdf is
\begin{displaymath}
\frac{e^{ax+b}}{C}.
\end{displaymath}
So the CDF
\begin{displaymath}
F'(x)=\int_{z_{j-1}}^{x'} \,  \frac{e^{ax+b}}{C} \, dx = \left. \frac{1}{ca}\, e^{ax+b} \, \right|_{z_{j-1}}^{x'} = \frac{ca}{e^b}\,(e^{ax'}-e^{az_{j-1}}).
\end{displaymath}
Then the inverse of $F(x')$ will be
\begin{displaymath}
\frac{\log\,(\frac{ac}{e^b}+e^{az_{j-1}})}{a}.
\end{displaymath}


In the special case of $a=0$ which means that for $x^* \in [z_{j-1}, z_j]$, $u(x) = b $ where $b=h(x_j)$. \\Then the pdf will be  be $ \frac{e^b}{c}x $ which is an uniform distribution in $[z_{j-1}, z_j]$. In this situation, we just sample $x^*$ from $unif \sim  (z_{j-1}, z_j)$.





\subsection*{ upper method }
In this method, we calculate the $u_{k}(x)$ for $x^*$ which we sample from the sampling method.


\subsection*{ lower method }
In this method, we calculate the $l_{k}(x)$ for $x^*$ which we sample from the sampling method.


\subsection*{ update method }
In this method, we test the sample element $x^{*}$ from sampling method. Once it is accepted, we add the $x^{*}$ into the set of outputs.

%%%%%%%%%%%%%%%

\section{Testing}

For testing, we rigorously tested each method to ensure it was functioning properly.  To test the overall function, we sampled over a few well known distributions with different bounds, including:  normal, gamma, truncated normal.  The results are shown below.\\
First we test the standard normal with varying bounds , the code to do so and the histogram of the resulting samples are given below:

\begin{lstlisting}[frame=single]
samples <- ars(n=10000,fx=function(x){(1/sqrt(2*pi)*exp((-(x-0)^2)/2))}, bounds=c(-Inf, Inf) )
sample<-ars(10000,function(x){(1/sqrt(2*pi)*exp((-(x-0)^2)/2))},c(-2, 2))
sample<-ars(10000,function(x){(1/sqrt(2*pi)*exp((-(x-0)^2)/2))},c(-Inf, 2))
sample<-ars(10000,function(x){(1/sqrt(2*pi)*exp((-(x-0)^2)/2))},c(-2, Inf))


\end{lstlisting}
\clearpage
\begin{figure}[htbp!]
 \centering
\caption{Histogram of samples from $f(x)=\frac{1}{\sqrt{2\pi}}e^{-\frac{x^2}{2}}$ with different bounds}
  \includegraphics[width=1.0\textwidth]{standardnormal}
\end{figure}


First we test the standard normal with varying bounds , the code to do so and the histogram of the resulting samples are given below:

\begin{lstlisting}[frame=single]
sample<-ars(10000,function(x){(10^10)/gamma(10)*x^9*exp(-10*x)},c(0.01, Inf))
sample<-ars(10000,function(x){10*exp(-10*x)},c(0, Inf))


\end{lstlisting}
\clearpage
\begin{figure}[htbp!]
 \centering
\caption{}
  \includegraphics[width=1.0\textwidth]{gammaexp}
\end{figure}




%%%%%%%%%%%%%%%

\section{Contributions}

Everyone in the group contributed relatively equally to each aspect of the project.  Some efforts were more focused as follows:

\subsection*{Code writing}

General structure: Lisa

Methods: James, Siwei, Hsin-Wei

\subsection*{Code testing}

Methods testing: James, Siwei

Overall function tests: Lisa, Hsin-Wei

\subsection*{Documentation} 

Manual creation: Lisa

Project write-up:  Hsin-Wei, Lisa

\end{document}


%\begin{lstlisting}
%x = 0:0.1:100;
%m = 0:19;
%
%y = zeros(1,1001);
%
%for i = 1:1001
%    xis = (0.05.*( 1 + x( i )  ) ).^m ;
%    y(i) = (((0.05.*( 1 + x( i )  ) ).^20 ) * ((1/x(i) + 10^4)/( 1/x(i) + 1 )))  /(sum( xis  ) + (((0.05.*( 1 + x( i )  ) ).^20 ) * ((1/x(i) + 10^4)/( 1/x(i) + 1 ))) );
%end
%
%plot( x, y)
%
%\end{lstlisting}

%\begin{figure}[htbp!]
%  \centering
%  \caption{ Plot of P(M*) versus x for strong binding}
%    \includegraphics[width=.73\textwidth]{chem220_ps7_4_2}
%\end{figure}


